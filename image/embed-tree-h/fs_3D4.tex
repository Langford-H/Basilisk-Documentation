\documentclass{article}
\usepackage{tikz}
\usepackage{tikz-3dplot}
\usepackage[active,tightpage]{preview}
\PreviewEnvironment{tikzpicture}
\setlength\PreviewBorder{0.125pt}
%
% File name: skew-polygon.tex
% Description: 
% A geometric representation of a skew polygon is shown.
% 
% Date of creation: October, 27th, 2021.
% Date of last modification: October, 9th, 2022.
% Author: Efra�n Soto Apolinar.
% https://www.aprendematematicas.org.mx/author/efrain-soto-apolinar/instructing-courses/
% Source: page 352 of the 
% Glosario Ilustrado de Matem\'aticas Escolares.
% https://tinyurl.com/5udm2ufy
% English version: 
% https://tinyurl.com/yayyxrbz
% 
% Terms of use: According to TikZ.net
% https://creativecommons.org/licenses/by-nc-sa/4.0/
% Your commitment to the terms of use is greatly appreciated.
%
\begin{document}
%
\tdplotsetmaincoords{75}{110}
%
\begin{tikzpicture}[tdplot_main_coords,scale=1.75]
	\pgfmathsetmacro{\a}{3.0} % length
	\pgfmathsetmacro{\h}{3.0}	% height
	\pgfmathsetmacro{\n}{100}	% number of intervals
	% Vertices of the box
	\coordinate (O) at (0,0,0);
	\coordinate (A) at (\a,0,0);
	\coordinate (B) at (\a,\a,0);
	\coordinate (C) at (0,\a,0);
	\coordinate (D) at (0,0,\h);
	\coordinate (E) at (\a,0,\h);
	\coordinate (F) at (\a,\a,\h);
	\coordinate (G) at (0,\a,\h);
        
        %Subcell points
        \coordinate (o) at (0,0,\a/2);
	\coordinate (a) at (0,\a,\a/2);
	\coordinate (b) at (\a,\a,\a/2);
	\coordinate (c) at (\a,0,\a/2);
 
	\coordinate (d) at (0,\a/2,0);
	\coordinate (e) at (0,\a/2,\a);
	\coordinate (f) at (\a,\a/2,\a);
	\coordinate (g) at (\a,\a/2,0);
 
 	\coordinate (O1) at (\a/2,\a/2,\a/2);
	% First I draw the box
	\draw[blue,thick,dashed] (O) -- (A) -- (B) -- (C) -- (O);	% plano z = 0
	\draw[blue,thick,dashed] (O) -- (D);	% Edge on the $z$ axis 
	\draw[blue,thick,dashed] (A) -- (E);	% vertical edge $(\a,0)$
	\draw[blue,thick,dashed] (C) -- (G);	% vertical edge $(0,\a)$
	% Edges of the external box
	\draw[blue,thick,dashed] (D) -- (E) -- (F) -- (G) -- (D);	% Plane $z = \h$
	\draw[blue,thick,dashed] (B) -- (F); % vertical edge $(\a,\a)$

        %horizontal subcells
        %middle
        \draw[blue,thick,dashed] (o) -- (a) -- (b) -- (c) -- cycle;
        \draw[blue,thick,dashed] (O1) --+ (0,-1.5,0) --+ (0,1.5,0);
        \draw[blue,thick,dashed] (O1) --+ (-1.5,0,0) --+ (1.5,0,0);
        
        \draw[blue,thick,dashed] ($(O1)+(0,0,1.5)$) --+ (0,-1.5,0) --+ (0,1.5,0);
        \draw[blue,thick,dashed] ($(O1)+(0,0,-1.5)$) --+ (0,-1.5,0) --+ (0,1.5,0);

        \draw[blue,thick,dashed] ($(O1)+(0,1.5,1.5)$) --+ (0,0,-3);
        \draw[blue,thick,dashed] ($(O1)+(0,-1.5,-1.5)$) --+ (0,0,3);
        
        %vertical subcells
        \draw[blue, thick, dashed] (d) -- (e) -- (f) -- (g) -- cycle;
        \draw[blue,thick,dashed] ($(O1)+(0,0,-1.5)$) --+ (0,0,3);
        \draw[blue,thick,dashed] (o) -- (a) -- (b) -- (c) -- cycle;

        \filldraw[fill=cyan, opacity=0.5, blue, thick , dashed](O)--(D)--(E)--(A)--cycle;
        \filldraw[fill=cyan, opacity=0.5, blue, thick , dashed](C)--(G)--(F)--(B)--cycle;

        %AXI
        \draw[->,thick](O1)node[below right]{$(0,0,0)$}--++(0,1.5,0)node[above right]{$(0.5,0,0)$}--++(0,0.5,0)node[below]{$X$};
        \draw[->,thick](O1)--++(0,0,1.5)--++(0,0,0.5)node[right]{$Y$};
        \draw[->,thick](O1)--++(1.5,0,0)--++(0.5,0,0)node[below]{$Z$};

        %highlight selecting cube
        \draw[red,thick] (O1) --++ (0,0,1.5) --++ (1.5,0,0) --++ (0,0,-1.5) -- cycle;
        \draw[red,thick] (O1) --++ (0,1.5,0) --++ (1.5,0,0) --++ (0,-1.5,0);
        \draw[red,thick] ($(O1)+(0,0,1.5)$) --++ (0,1.5,0) --++ (1.5,0,0) --++ (0,-1.5,0);
        \draw[red,thick] ($(O1)+(0,1.5,0)$) --++ (0,0,1.5);
        \draw[red,thick] ($(O1)+(1.5,1.5,0)$) --++ (0,0,1.5);

\end{tikzpicture}
%	
\end{document}