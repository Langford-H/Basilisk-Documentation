\chapter{centered.h Documentation}
\ifsingle
\maketitle
\fi
\chaptermeta[1.0][2025-01-01]

\section{Introduction}
This solver addresses the incompressible Navier-Stokes equations\cite{2003_Popinet,2015_Popinet}:

\begin{gather}
    \frac{\partial \mathbf{u}}{\partial t} 
    + \nabla \cdot (\mathbf{u} \otimes \mathbf{u}) 
    = \frac{1}{\rho} \left(-\nabla p + \nabla \cdot (2\mu \mathbf{D})\right) 
    + \mathbf{a}
    \label{equ:centered-nseq1} \\
    \nabla \cdot \mathbf{u} = 0 
    \label{equ:centered-nseq2}
\end{gather}

where the deformation tensor is defined as 
$
\mathbf{D} = \frac{1}{2} (\nabla \mathbf{u} + (\nabla \mathbf{u})^T)
$.

For multiphase flows, the acceleration term $\mathbf{a}$ includes the effects of interfacial forces, such as surface tension. With the incompressibility constraint in equation \ref{equ:centered-nseq2}, the velocity $\mathbf{u}$ and pressure gradient $\nabla p$ at the next time step can be determined using approximate projection methods. The discrete form is:

\begin{gather}
    \frac{\mathbf{u}^{n+1} - \mathbf{u}^n}{\Delta t} 
    + \nabla \cdot (\mathbf{u} \otimes \mathbf{u})^{n+1/2} 
    = \frac{1}{\rho^{n+1/2}} \left(-\nabla p^{n+1} 
    + \nabla \cdot \left(2\mu^{n+1/2} \mathbf{D}^{**}\right)\right) 
    + \mathbf{a}^{n+1/2}
    \label{equ:centered-nseqdes1} \\
    \nabla \cdot \mathbf{u}^{n+1} = 0 \label{equ:centered-nseqdes2}
\end{gather}

Here, the viscosity term is computed implicitly using the intermediate velocity $\mathbf{u}^{**}$.

The following subsections will provide a theoretical overview of each step, with practical program analysis presented in subsequent sections. Notably, in the following subsections, face-centered variables will be denoted by the subscript $f$, such as $\mu_f$, to distinguish them from cell-centered variables. Readers are also referred to the excellent introduction written by Edoardo Cipriano \cite{2024_Cipriano}, which has been a great inspiration for this documentation.

\subsection{Algorithm Overview}
To summarize, solving the Navier-Stokes equations involves four main steps: \textbf{Property Update, Advection, Viscosity, and Projection}, corresponding to distinct stages of the solution process. The first three steps involve solving the following equations:

\begin{gather}
    \frac{c^{n+1/2} - c^{n-1/2}}{\Delta t} 
    + \mathbf{F}(c,\mathbf{u}_f^n) = 0 \label{equ:centered-colorfunc} \\
    \rho^{n+1/2} \left[ \frac{\mathbf{u}^{**} - \mathbf{u}^n}{\Delta t} 
    + \nabla \cdot (\mathbf{u} \otimes \mathbf{u})^{n+1/2} 
    - \mathbf{a}^{n-1/2} 
    + \frac{\nabla p^n}{\rho^{n-1/2}} \right] 
    = \nabla \cdot \left( 2\mu^{n+1/2}_f \mathbf{D}^{**} \right)\label{equ:centered-advdiff}\\
    \mathbf{u}^{***} = \mathbf{u}^{**} 
    - \Delta t \left( \mathbf{a}^{n-1/2} 
    - \frac{\nabla p^n}{\rho^{n-1/2}} \right)\label{equ:centered-diffalter}
\end{gather}

Once $\mathbf{u}^{***}$ is computed, we obtain $\mathbf{u}^{n+1}$ by:

\begin{equation}
    \mathbf{u}^{n+1} = \mathbf{u}^{***} 
    + \Delta t \left( \mathbf{a}^{n+1/2} 
    - \frac{\nabla p^{n+1}}{\rho^{n+1/2}} \right)
\end{equation}

Using the incompressibility constraint:

\begin{gather}
    \nabla \cdot \mathbf{u}^{n+1} = 0
\end{gather}

the pressure gradient $\nabla p^{n+1}$ can be computed by solving a Poisson equation in the \textbf{Projection} step.

\subsubsection{Property Update}
Equation \ref{equ:centered-colorfunc} is the discrete form of advection equation for markers
\begin{equation}
    \frac{\partial c}{\partial t} + \nabla\cdot(c\mathbf{u}_f^n) = 0
\end{equation}
the scheme of advection term $\mathbf{F}(c,\mathbf{u}_f^n)$ varies depending on the practical condition, the $\mathbf{u}_f^n$ here is the strict non-divergence face velocity obtained in the last final step (see section \ref{sec:centered-celltoface}). If $c$ denotes the volume of fluids and the header file \texttt{two-phase.h} is included, the geometric advection scheme in \texttt{vof.h} is activated. Otherwise, if $c$ represents tracers, the BCG scheme is applied to solve the equation. For further details, refer to the \texttt{tracers.h} and \texttt{vof.h} documentation.

The density $\rho$ and viscosity $\mu_f$ at the $n+1/2$ time step are determined based on the distribution of $c$, this step is implemented in \texttt{two-phase-generic.h}. For those condition with large density ratio, there is option to smear the $\alpha_f=1/\rho$ in the same header file, as it is considered in poisson equation solver. All these steps are achieved by event succession, note although the body force such as surface tension is associated with distribution of $c$, it is actually implemented in the \textbf{Projection} step also through succession of event `acceleration'.
\subsection{Advection}

In this step, we merge the first two terms on the L.H.S. of Equation \ref{equ:centered-advdiff}, solving:  
\begin{equation}
    \rho^{n+1/2}\left[\frac{\mathbf{u}^* - \mathbf{u}^n}{\Delta t} + \nabla\cdot(\mathbf{u}\otimes\mathbf{u})^{n+1/2}\right] = 0\label{equ:centered-adv}
\end{equation}  
where $\mathbf{u}^*$ is the intermediate velocity. Given the face velocity $\mathbf{u}^{n+1/2}_f$, this equation is solved using functions provided by the header file \texttt{bcg.h}, following the BCG scheme.  For further details on the BCG scheme, please refer to the \texttt{bcg.h} documentation.\\
The face velocity $\mathbf{u}^{n+1/2}_f$ is first computed using the BCG scheme and then projected to satisfy the non-divergence constraint, as required by the functions provided in \texttt{bcg.h}. Since these functions do not directly output values at $n+1/2$, this computation is performed separately in \texttt{centered.h}.\\

\subsection{Diffusion}
This step aims to compute $\mathbf{u}^{***}$.  
Starting from Equation \ref{equ:centered-adv}, Equation \ref{equ:centered-advdiff} becomes:  
\begin{equation}
    \rho^{n+1/2} \left[ \frac{\mathbf{u}^{**} - \mathbf{u}^*}{\Delta t}  
    - \mathbf{a}^{n-1/2} 
    + \frac{\nabla p^n}{\rho^{n-1/2}} \right] 
    = \nabla \cdot \left( 2\mu^{n+1/2}_f \mathbf{D}^{**} \right)\label{equ:centered-diff}
\end{equation}  
Here, $\mathbf{u}^{**}$ is computed implicitly, and $\mathbf{u}^{***}$ is subsequently determined by equation \ref{equ:centered-diffalter}.

\subsection{Projection}

The final step enforces the non-divergence constraint. A key challenge is managing the conversion between cell-centered and face-centered values, as velocities are stored at cell centers, accelerations are stored at cell faces, and the projection function operates on face values.  

\subsubsection{From Cell-Centered to Face-Centered: Computing $p^{n+1}$}\label{sec:centered-celltoface}  

We compute the pressure $p^{n+1}$ from:  
\begin{gather}
    \frac{\mathbf{u}_f^{n+1} - \mathbf{u}_f^{***}}{\Delta t}
  = -\frac{\nabla p^{n+1}}{\rho^{n+1/2}} + \mathbf{a}_f^{n+1/2}\\
    \nabla\cdot\mathbf{u}_f^{n+1} = 0
\end{gather}  
which simplifies to:  
\begin{equation}
    \nabla\left(\frac{\nabla p^{n+1}}{\rho^{n+1/2}}\right) = \nabla\left(\frac{\mathbf{u}_f^{***}}{\Delta t} + \mathbf{a}_f^{n+1/2}\right)
\end{equation}  
Using this relationship, the cell-centered pressure is computed with a multigrid solver applied to the face velocity:  
\begin{equation}
    \mathbf{u}_f^{****} = \frac{\mathbf{u}^{***}[] + \mathbf{u}^{***}[-1]}{2}
  + \Delta t \mathbf{a}_f^{n+1/2}
\end{equation}  
The divergence-free face velocity $\mathbf{u}_f^{n+1}$ is updated for computing the advection of tracers in the next time step.  

\subsubsection{From Face-Centered to Cell-Centered: Updating $\mathbf{u}^{n+1}$}  

With $p^{n+1}$ computed, the cell-centered velocity $\mathbf{u}^{n+1}$ is updated:  
\begin{equation}
    \frac{\mathbf{u}^{n+1} - \mathbf{u}^{***}}{\Delta t} = -\frac{\nabla
    p^{n+1}}{\rho^{n+1/2}} + \mathbf{a}^{n+1/2}
\end{equation}  
To transform face-centered values to cell-centered values, the face-centered pressure gradient is computed and combined with the face-centered acceleration $\mathbf{a}_f^{n+1/2}$ to yield:  
\begin{equation}
    \mathbf{g}_f^{n+1} = \mathbf{a}_f^{n+1/2} + \frac{\nabla p^{n+1}}{\rho_f^{n+1/2}}
\end{equation}  
The cell-centered $\mathbf{g}^{n+1}$ is obtained by averaging:  
\begin{equation}
    \mathbf{g}^{n+1} = \frac{\mathbf{g}_f^{n+1}[1] + \mathbf{g}_f^{n+1}[]}{2}
\end{equation}  
Finally, the velocity is updated:  
\begin{equation}
    \mathbf{u}^{n+1} = \mathbf{u}^{***} + \Delta t \mathbf{g}^{n+1}
\end{equation}  

\subsection{Notes on the Projection Method and VOF-NS Equations}
\subsubsection{Approximate vs. Exact Projection}  
The approximate projection method, as initially described, pertains to an algorithm where the discrete Laplace operator does not precisely equal the discrete divergence of the discrete gradient, according to Almgren (2000)\cite{2000_Almgren}. This situation arises exclusively on grids where only the cell-center is active, due to the misalignment caused by the face-centered vector. However, \texttt{Basilisk} circumvents this issue. Nonetheless, as previously discussed, while the face-centered velocity \para{$u_f$} maintains strict non-divergence, the cell-centered velocity \para{$u_c$} does not, as it is derived by subtracting the average of the updated pressure-acceleration at the face center. Consequently, this is why the projection method utilized by \texttt{Basilisk} is termed an approximate projection.

\subsubsection{The Non-conserving Form of the VOF-NS Equations}
When applying the VOF method, equation \ref{equ:centered-nseq1} cannot be directly obtained, since the density is no longer a constant but a time-dependent function, given by
\begin{equation}
    \rho(\mathbf{x}, t) = \rho_1 c^n + \rho_2 (1 - c^n).
\end{equation}
Consequently, the incompressible Navier--Stokes equations take the form
\begin{gather}
    \frac{\partial (\rho \mathbf{u})}{\partial t} 
    + \nabla \cdot (\rho \mathbf{u} \otimes \mathbf{u}) 
    = -\nabla p + \nabla \cdot (2 \mu \mathbf{D}) 
    + \mathbf{a}, \label{equ:centered-incnseq1} \\
    \frac{\partial \rho}{\partial t} + \nabla \cdot (\rho \mathbf{u}) = 0, \label{equ:centered-incnseq2} \\
    \nabla \cdot \mathbf{u} = 0. \label{equ:centered-incnseq3}
\end{gather}
Here, the mass continuity equation \ref{equ:centered-incnseq2} and the incompressibility constraint \ref{equ:centered-incnseq3} are decoupled for the same reason.  
By substituting the incompressibility condition back into the continuity equation, we obtain
\begin{equation}
    \frac{\partial \rho}{\partial t} + \mathbf{u} \cdot \nabla \rho = 0. \label{equ:centered-incnseq4}
\end{equation}
Combining equation \ref{equ:centered-incnseq4} with the momentum equation \ref{equ:centered-incnseq1}, we finally recover the Navier--Stokes equations in the form of equation \ref{equ:centered-nseq1}.  
This formulation allows us to avoid explicitly handling the highly discontinuous advection terms that arise under large density ratios, such as in the case of an air--water interface ($\rho_\text{water} : \rho_\text{air} \sim 1000:1$).

\printbibliography
